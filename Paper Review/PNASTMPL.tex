%%%%%%%%%%%%%%%%%%%%%%%%%%%%%%%%%%%%%%%%%
% Proceedings of the National Academy of Sciences (PNAS)
% LaTeX Template
% Version 1.0 (19/5/13)
%
% This template has been downloaded from:
% http://www.LaTeXTemplates.com
%
% Original author:
% The PNAStwo class was created and is owned by PNAS:
% http://www.pnas.org/site/authors/LaTex.xhtml
% This template has been modified from the blank PNAS template to include
% examples of how to insert content and drastically change commenting. The
% structural integrity is maintained as in the original blank template.
%
% Original header:
%% PNAStmpl.tex
%% Template file to use for PNAS articles prepared in LaTeX
%% Version: Apr 14, 2008
%
%%%%%%%%%%%%%%%%%%%%%%%%%%%%%%%%%%%%%%%%%

%----------------------------------------------------------------------------------------
%	PACKAGES AND OTHER DOCUMENT CONFIGURATIONS
%----------------------------------------------------------------------------------------

%------------------------------------------------
% BASIC CLASS FILE
%------------------------------------------------

%% PNAStwo for two column articles is called by default.
%% Uncomment PNASone for single column articles. One column class
%% and style files are available upon request from pnas@nas.edu.

%\documentclass{pnasone}
\documentclass{pnastwo}

%------------------------------------------------
% POSITION OF TEXT
%------------------------------------------------

%% Changing position of text on physical page:
%% Since not all printers position
%% the printed page in the same place on the physical page,
%% you can change the position yourself here, if you need to:

% \advance\voffset -.5in % Minus dimension will raise the printed page on the 
                         %  physical page; positive dimension will lower it.

%% You may set the dimension to the size that you need.

%------------------------------------------------
% GRAPHICS STYLE FILE
%------------------------------------------------

%% Requires graphics style file (graphicx.sty), used for inserting
%% .eps/image files into LaTeX articles.
%% Note that inclusion of .eps files is for your reference only;
%% when submitting to PNAS please submit figures separately.

%% Type into the square brackets the name of the driver program 
%% that you are using. If you don't know, try dvips, which is the
%% most common PC driver, or textures for the Mac. These are the options:

% [dvips], [xdvi], [dvipdf], [dvipdfm], [dvipdfmx], [pdftex], [dvipsone],
% [dviwindo], [emtex], [dviwin], [pctexps], [pctexwin], [pctexhp], [pctex32],
% [truetex], [tcidvi], [vtex], [oztex], [textures], [xetex]

\usepackage{graphicx}

%------------------------------------------------
% OPTIONAL POSTSCRIPT FONT FILES
%------------------------------------------------

%% PostScript font files: You may need to edit the PNASoneF.sty
%% or PNAStwoF.sty file to make the font names match those on your system. 
%% Alternatively, you can leave the font style file commands commented out
%% and typeset your article using the default Computer Modern 
%% fonts (recommended). If accepted, your article will be typeset
%% at PNAS using PostScript fonts.

% Choose PNASoneF for one column; PNAStwoF for two column:
%\usepackage{PNASoneF}
%\usepackage{PNAStwoF}

%------------------------------------------------
% ADDITIONAL OPTIONAL STYLE FILES
%------------------------------------------------

%% The AMS math files are commonly used to gain access to useful features
%% like extended math fonts and math commands.

\usepackage{amssymb,amsfonts,amsmath}

%------------------------------------------------
% OPTIONAL MACRO FILES
%------------------------------------------------

%% Insert self-defined macros here.
%% \newcommand definitions are recommended; \def definitions are supported

%\newcommand{\mfrac}[2]{\frac{\displaystyle #1}{\displaystyle #2}}
%\def\s{\sigma}

%------------------------------------------------
% DO NOT EDIT THIS SECTION
%------------------------------------------------

%% For PNAS Only:
\contributor{Submitted to Proceedings of the National Academy of Sciences of the United States of America}
\url{www.pnas.org/cgi/doi/10.1073/pnas.0709640104}
\copyrightyear{2017}
\issuedate{December 9th 2017}
\volume{Volume}
\issuenumber{Issue Number}

%----------------------------------------------------------------------------------------

\begin{document}

%----------------------------------------------------------------------------------------
%	TITLE AND AUTHORS
%----------------------------------------------------------------------------------------

\title{Review on ``The best privacy defense is a good privacy offense: obfuscating a search engine user's profile''} % For titles, only capitalize the first letter

%------------------------------------------------

%% Enter authors via the \author command.  
%% Use \affil to define affiliations.
%% (Leave no spaces between author name and \affil command)

%% Note that the \thanks{} command has been disabled in favor of
%% a generic, reserved space for PNAS publication footnotes.

%% \author{<author name>
%% \affil{<number>}{<Institution>}} One number for each institution.
%% The same number should be used for authors that
%% are affiliated with the same institution, after the first time
%% only the number is needed, ie, \affil{number}{text}, \affil{number}{}
%% Then, before last author ...
%% \and
%% \author{<author name>
%% \affil{<number>}{}}

%% For example, assuming Garcia and Sonnery are both affiliated with
%% Universidad de Murcia:
%% \author{Roberta Graff\affil{1}{University of Cambridge, Cambridge,
%% United Kingdom},
%% Javier de Ruiz Garcia\affil{2}{Universidad de Murcia, Bioquimica y Biologia
%% Molecular, Murcia, Spain}, \and Franklin Sonnery\affil{2}{}}

\author{Joshua Fenech\affil{1}{Jean Monnet University},
\and
Omar Elsabrout\affil{1}{University of Oregon}}

\contributor{Submitted to the faculty of science and technology at Jean Monnet University}

%----------------------------------------------------------------------------------------

\maketitle % The \maketitle command is necessary to build the title page

\begin{article}

%----------------------------------------------------------------------------------------
%	ABSTRACT, KEYWORDS AND ABBREVIATIONS
%----------------------------------------------------------------------------------------

\begin{abstract}
Since the growth rate of the Internet is almost exponential, it has a drastic effect on people's lives everyday. Such effect is accompanied with many issues such as user privacy. It is not settled until now how service providers are allowed to use stored information of the users and compromise their privacy. J\"{o}rg Wicker and Stefan Kramer introduce a tool that utilizes machine learning and data mining to confuse search engines to protect the user's privacy by obfuscating exploited personal information. Not only methods are introduced for such technique, but also an experiment to evaluate its results indicating whether this approach should be investigated further.
\end{abstract}

%------------------------------------------------

%\keywords{Keyword1 | Keyword2 | Keyword3} % When adding keywords, separate each term with a straight line: |

%------------------------------------------------

%% Optional for entering abbreviations, separate the abbreviation from
%% its definition with a comma, separate each pair with a semicolon:
%% for example:
%% \abbreviations{SAM, self-assembled monolayer; OTS,
%% octadecyltrichlorosilane}

% \abbreviations{}
%\abbreviations{SAM, self-assembled monolayer; OTS, octadecyltrichlorosilane}

%----------------------------------------------------------------------------------------
%	PUBLICATION CONTENT
%----------------------------------------------------------------------------------------

%% The first letter of the article should be drop cap: \dropcap{} e.g.,
%\dropcap{I}n this article we study the evolution of ''almost-sharp'' fronts

\section{Introduction}

\dropcap{T}he motivation behind providing this approach is that the current privacy protection have major flaws. On one hand, users rely on service providers to process their data, on the other hand, providers do not have any advantage in user privacy preserving technologies as the analysis of this data and sharing it with advertisers is the basis of their business model. No doubt that providers such as search engines must store information on users. Nevertheless, these data can be used to generate detailed profiles on a large scale and identify unidentified users. As mentioned, J\"{o}rg Wicker and Stefan Kramer target privacy from another perspective. Hence, they suggest a user tools to defend her or his privacy so the user does not have to rely on the other uncontrollable side for this issue. Data is conventionally stored in large scales
and analyzed automatically using data mining technologies. As a result, the intuitive approach to protect the user’s private information would be to flood the data storage with random data and hope the user’s interest or identity would be obfuscated. On the contrary, data mining algorithms are designed to distinguish a signal from random noise. Consequently, this approach will fail in most settings where the data is analyzed with sophisticated data mining algorithms. The paper tackles this issue in a more highly developed manner as it gives a brief discussion of the user and search engine model in addition to the proposed method. Moreover, it shows the details of the experimental set-up and results. Taking into account, the paper should be considered as proof-of-concept and not a final product ready for the market. In this review, we present pointers for the given approach and criticize it by explaining its strong arguments and also its weak ones. This is not meant to be a summary of the tackled paper as we merely mention concepts and do not dive into details.\cite{DBLP:journals/datamine/WickerK17}. 

%------------------------------------------------

\section{Discussion}

One of the first obstacles that is faced and discussed in the paper, is the user model. The authors admit a problem of major simplification for the user's interest categories as a result of reducing these categories to one at a time. The reason behind such simplification is to simplify the evaluation so they can compare the
reaction of the search engine to one interest category. Although they stated that future work will address users with multiple interest categories and users with variably strong interest in multiple categories, we think that such simplification needs to be accompanied with assurances of the ability to upscale the model for multiple categories or at lease some pointers for such ability. These assurances are not provided which leaves the problem unattended. Nonetheless, another simplification is justified which is to only address query results that provided ads due to the problem's nature.

Another obstacle that might be rather unavoidable is that obfuscating a search engine user's profile, even though if it protects the user's privacy, reduces the user's experience quality. A user who utilizes this approach will lose most of personalized services and automated customizations provided by search engines if not all of them. We know that this is a decision that users must make. However, it is a hard one as these services are more practical and useful so the trade-off might not be fair. The author's assumption that user prefers privacy overlooks the idea that the number of those users can be low in comparison with the ones who prefer personalization. Despite that, we think that it is still perfect to execute the obfuscation idea in form of a tool that can be easily deactivated if the user wanted. This does not affect the user's profile if he or she wishes to return to using the search engines normally. 

In addition, obfuscating search engines and service providers waste their resources and invalidates their research results and statistical studies that might lead to new applications based on the user's needs. These needs are usually investigated by online surveys and mining user data which will be useless because of mining fake or randomized data. 

%------------------------------------------------

\section{Method}
Since the full knowledge of the states and information of the search engine are not available, the algorithm behaves according to only the feedback from the search engine in the form of ads. The method of such approach is to based on text ads, which is certainly a limitation as mass media now depends on many more forms of advertisement that involve audio and video. Nonetheless, it is considered a great start though and a milestone for promising future work. Service provider such as search engines display targeted ads depending on the submitted query by the user which must go through an interest category process to categorize the query. In addition, displayed ads also depend on the user's profile. This plays a greater role specially for advertisers as dedicating ads to targeted audience is more efficient and promises perfecting the selling point.

For such process to succeed, ads must be processed and assigned to categories in advance to be displayed to their matching needs. Not only ads are assigned, but also users are assigned to categories depending on their profiles either by their user accounts or by artificial ones depending on IP tracking and cookies.

Revisiting the concept of privacy defense which is the main purpose of J\"{o}rg Wicker's and Stefan Kramer's method, privacy metrics and standards must be presented to evaluate such method. For instance, an objective function is presented in the paper for such purpose \eqref{eq1}.

\begin{equation}
\sigma(\kappa_i, P) = \sum_{p_j\in P, \kappa_j\in T}^{} p_j d_T(\kappa_i, \kappa_j)
\label{eq1}
\end{equation}

%Referencing equation \eqref{qg1}.

According to this notation, the user's interest category is expressed with $\kappa$ and these categories are predefined by the search engine. Moreover, $P$ represents the distribution of probabilities of the categories, $T$ represents the category tree and $d_T$ is the tree distance. This objective function is explained in details in the original paper. However, our report is not interested in such redundancy. Our interest resides in the significance of $\sigma$ which represents the score or the distance between the user's current interest category and the assigned category to the user. This $\sigma$ shows how the approach appears to be dynamic and it also shows the space margin of the algorithm to learn from the search engine's response to ad displaying.

The second component that is required for privacy defense is using the feedback of ads. This opens the discussion of category prediction of ads because search engines provide example queries for each category. These example queries or sample queries of category trees are the main input to train classifiers for ad categorization. Not only one classifier in this case, but as many classifiers as the number of available categories as every category has its own classifier. After certain training period, these classifiers can be used for more than classifying queries. Also, they can be used to classify any plain text which enables processing all text-based ads. As a result, prediction on ads works very well thanks to the similarities of structure between queries and text-based ads.

%------------------------------------------------

\section{Experiment and Results}

Considering the environment of the presented experiment in the paper, we find that it was redundant to execute the experiment with submitting queries from random categories as it does not provide relevant results and does not contribute in the significance of the privacy offense method. On the contrary, submitting queries from the category that is the furthest away from the user's interest category is more relevant to the privacy offense method \ref{fig:results}.

\begin{figure}[h]
\includegraphics[scale=0.3]{Figure}
\caption{Results}
\label{fig:results}
\centering
\end{figure}

According to Wicker's experiment, all users submit random queries from their batch of interest categories in 10\% of the cases. These outputs are based on using twenty categories in the experiment environment which we believe is sufficient in the context of an experiment. Upscaling the method would actually lead to more issues with greater number of users and interest categories. Despite that, it was fairly mentioned in the future work section that it is a major concern to continue working on the method and crucial to its development. 

%------------------------------------------------

\section{Conclusion}

To sum up, the experiment is successful to a certain extent with respect to the assumed limitations. Nevertheless, we still need further research and more elaborate experiment environments.

There are more areas that deserve studying and should be tackled during the development of such tool like the fact that models of search engines also rely on time and date of the query which is a factor that should be considered and taken advantage of. At the end of the day, the major simplification of learning for only one interest is still the major concern and the greatest limitation of the tool. Future work is interesting as it can involve more feedback than just ads from search engines but also use more features that they provide to predict their models and reactions. This could lead to a more sophisticated tool and could provide a better privacy defense model.

In this report, we discussed the main points of interest in the research paper while criticizing its weak and strong arguments. We started by the motivation behind the research in the introduction. Then, we presented how the discussion in the paper had strong arguments but with a few flaws. Afterwards, we discussed the method itself and how is it modeled which is the most important and interesting part of the paper. Last but not least, we mentioned how the results of the experiment and their environment contribute to the results and how they can improve.  

%----------------------------------------------------------------------------------------

%% Optional Materials and Methods Section
%% The Materials and Methods section header will be added automatically.

%\begin{materials}
%Suspendisse viverra eleifend nulla at facilisis. Nullam eget tellus orci. Cras sit amet lorem velit. Maecenas rhoncus pellentesque orci eget vulputate. Phasellus massa nisi, mattis nec elementum accumsan, blandit non neque. In ac enim elit, sit amet luctus ante. Cras feugiat commodo lectus, vitae convallis dui sagittis id. In in tellus lacus, sed lobortis eros. Phasellus sit amet eleifend velit. Duis ornare dapibus porttitor. Maecenas eros velit, dignissim at egestas in, tincidunt lacinia erat. Proin elementum mi vel lectus suscipit fringilla. Mauris justo est, ullamcorper in rutrum interdum, accumsan eget mi. Maecenas ut massa aliquet purus eleifend vehicula in a nisi. Fusce molestie cursus lacinia.
%
%\begin{definition}
%A bounded function $\theta$ is a weak solution of QG if for any
%$\phi\,\epsilon\,
%C_0^{\infty}(\fdb\times\mathbb{R}\times[0,\vep])$ we have
%\begin{eqnarray}
%&&  \int_{\mathbb{R}^+\times\fd\times\mathbb{R}} \hspace{-25pt}
% \theta(x,y,t)\, \pr_t \phi
%\,(x,y,t) dy dx dt+\nonumber\\
%  & +&\int_{\mathbb{R}^+\times\fd\times\mathbb{R}}
%\hspace{-26pt} \theta\,(x,y,t) u(x,y,t)\cdot\nabla\phi\,(x,y,t)
%dydxdt = 0 \label{weaksol} \end{eqnarray}
%where $u$ is determined previously.
%\end{definition}
%
%Vestibulum ante ipsum primis in faucibus orci luctus et ultrices posuere cubilia Curae; Mauris eu sapien nunc, sit amet accumsan dui. Nulla ac diam ut nunc placerat semper eget et libero. Vestibulum ante ipsum primis in faucibus orci luctus et ultrices posuere cubilia Curae; Cras hendrerit ullamcorper sapien vitae luctus. Quisque vel diam massa. Vestibulum dui nibh, facilisis vel vestibulum eu, viverra in quam.
%
%\begin{theorem}
%If the active scalar $\theta$ satisfies
%the equation \eqref{weaksol}, then $\varphi$ satisfies the equation
%\begin{eqnarray}
%\mfrac{\pr \varphi}{\pr t}(x,t)&=&\hspace{-2pt}\dst
%\int_{\fd}\mfrac{\mfrac{\pr \varphi}{\pr x}(x,t)-\mfrac{\pr
%\varphi}{\pr
%u}(u,t)}{[(x-u)^{2}+(\varphi(x,t)-\varphi(u,t))^{2}]^{\f12}}\nonumber\\
%&&
%\chi(x-u,\varphi(x,t)-\varphi(u,t)) du \hspace{3pt} +
%\nonumber\\
%&+&\dst \int_{\fd} \Big{[}\mfrac{\pr \varphi} {\pr
%x}(x,t)-\mfrac{\pr \varphi}{\pr u} (u,t)\Big{]}
%\nonumber\\&&
%\eta(x-u,\varphi(x,t)-\varphi(u,t)) du + Error
%\end{eqnarray}
%with $|Error|\leq C\, \delta | log\delta| $ where $C$ depends only
%on $\|\theta\|_{L^{\infty}}$ and $\|
%\nabla\varphi\|_{L^{\infty}}$.
%\end{theorem}
%
%Class aptent taciti sociosqu ad litora torquent per conubia nostra, per inceptos himenaeos. Integer accumsan ornare tortor at varius. Phasellus ullamcorper blandit dolor sit amet tempus. Curabitur ligula urna, ultrices in iaculis eu, eleifend vel urna. Praesent ullamcorper imperdiet purus, ut interdum sem interdum dictum. Proin euismod volutpat eros ac mattis. Quisque sit amet massa ac tortor cursus malesuada at vitae nisi. Nam quis neque et nunc vehicula cursus sit amet at tellus.
%\end{materials}

%----------------------------------------------------------------------------------------
%	APPENDICES (OPTIONAL)
%----------------------------------------------------------------------------------------

%\appendix
%An appendix without a title.

%\appendix[Appendix title]
%An appendix with a title.

%----------------------------------------------------------------------------------------
%	ACKNOWLEDGEMENTS
%----------------------------------------------------------------------------------------

\begin{acknowledgments}
This work was part of an assignment for a research methodology course.
\end{acknowledgments}

%----------------------------------------------------------------------------------------
%	BIBLIOGRAPHY
%----------------------------------------------------------------------------------------

%% PNAS does not support submission of supporting .tex files such as BibTeX.
%% Instead all references must be included in the article .tex document. 
%% If you currently use BibTeX, your bibliography is formed because the 
%% command \verb+\bibliography{}+ brings the <filename>.bbl file into your
%% .tex document. To conform to PNAS requirements, copy the reference listings
%% from your .bbl file and add them to the article .tex file, using the
%% bibliography environment described above.  

%%  Contact pnas@nas.edu if you need assistance with your
%%  bibliography.

% Sample bibliography item in PNAS format:
%% \bibitem{in-text reference} comma-separated author names up to 5,
%% for more than 5 authors use first author last name et al. (year published)
%% article title  {\it Journal Name} volume #: start page-end page.
%% ie,
% \bibitem{Neuhaus} Neuhaus J-M, Sitcher L, Meins F, Jr, Boller T (1991) 
% A short C-terminal sequence is necessary and sufficient for the
% targeting of chitinases to the plant vacuole. 
% {\it Proc Natl Acad Sci USA} 88:10362-10366.


%% Enter the largest bibliography number in the facing curly brackets
%% following \begin{thebibliography}

%\begin{thebibliography}{10}
\bibliographystyle{plain}
\bibliography{bib}

%\bibitem{BN}
%M.~Belkin and P.~Niyogi, {\em Using manifold structure for partially
%  labelled classification}, Advances in NIPS, 15 (2003).

%\bibitem{BBG:EmbeddingRiemannianManifoldHeatKernel}
%P.~B\'erard, G.~Besson, and S.~Gallot, {\em Embedding {R}iemannian
%  manifolds by their heat kernel}, Geom. and Fun. Anal., 4 (1994),
%  pp.~374--398.
%
%\bibitem{CLAcha1}
%R.R.~Coifman and S.~Lafon, {\em Diffusion maps}, Appl. Comp. Harm. Anal.,
%  21 (2006), pp.~5--30.
%
%\bibitem{DiffusionPNAS}
%R.R.~Coifman, S.~Lafon, A.~Lee, M.~Maggioni, B.~Nadler, F.~Warner, and
%  S.~Zucker, {\em Geometric diffusions as a tool for harmonic analysis and
%  structure definition of data. {P}art {I}: Diffusion maps}, Proc. of Nat.
%  Acad. Sci.,  (2005), pp.~7426--7431.
%
%\bibitem{Clementi:LowDimensionaFreeEnergyLandscapesProteinFolding}
%P.~Das, M.~Moll, H.~Stamati, L.~Kavraki, and C.~Clementi, {\em
%  Low-dimensional, free-energy landscapes of protein-folding reactions by
%  nonlinear dimensionality reduction}, P.N.A.S., 103 (2006), pp.~9885--9890.
%
%\bibitem{DoGri}
%D.~Donoho and C.~Grimes, {\em Hessian eigenmaps: new locally linear
%  embedding techniques for high-dimensional data}, Proceedings of the National
%  Academy of Sciences, 100 (2003), pp.~5591--5596.
%
%\bibitem{DoGri:WhenDoesIsoMap}
%D.~L. Donoho and C.~Grimes, {\em When does isomap recover natural
%  parameterization of families of articulated images?}, Tech. Report Tech. Rep.
%  2002-27, Department of Statistics, Stanford University, August 2002.
%
%\bibitem{GruterWidman:GreenFunction}
%M.~Gr\"uter and K.-O. Widman, {\em The {G}reen function for uniformly
%  elliptic equations}, Man. Math., 37 (1982), pp.~303--342.
%
%\bibitem{Simon:NeumannEssentialSpectrum}
%R.~Hempel, L.~Seco, and B.~Simon, {\em The essential spectrum of neumann
%  laplacians on some bounded singular domains}, 1991.
%
%\bibitem{1}
%Kadison, R.\ V.\ and Singer, I.\ M.\ (1959)
%Extensions of pure states, {\it Amer.\ J.\ Math.\ \bf
%81}, 383-400.
%
%\bibitem{2}
%Anderson, J.\ (1981) A conjecture concerning the pure states of
%$B(H)$ and a related theorem. in {\it Topics in Modern Operator
%Theory}, Birkha\"user, pp.\ 27-43.
%

%\end{thebibliography}

%----------------------------------------------------------------------------------------

\end{article}

%----------------------------------------------------------------------------------------
%	FIGURES AND TABLES
%----------------------------------------------------------------------------------------

%% Adding Figure and Table References
%% Be sure to add figures and tables after \end{article}
%% and before \end{document}

%% For figures, put the caption below the illustration.
%%
%% \begin{figure}
%% \caption{Almost Sharp Front}\label{afoto}
%% \end{figure}

%\begin{figure}[h]
%\centerline{\includegraphics[width=0.4\linewidth]{placeholder.jpg}}
%\caption{Figure caption}\label{placeholder}
%\end{figure}

%% For Tables, put caption above table
%%
%% Table caption should start with a capital letter, continue with lower case
%% and not have a period at the end
%% Using @{\vrule height ?? depth ?? width0pt} in the tabular preamble will
%% keep that much space between every line in the table.

%% \begin{table}
%% \caption{Repeat length of longer allele by age of onset class}
%% \begin{tabular}{@{\vrule height 10.5pt depth4pt  width0pt}lrcccc}
%% table text
%% \end{tabular}
%% \end{table}

%\begin{table}[h]
%\caption{Table caption}\label{sampletable}
%\begin{tabular}{l l l}
%\hline
%\textbf{Treatments} & \textbf{Response 1} & \textbf{Response 2}\\
%\hline
%Treatment 1 & 0.0003262 & 0.562 \\
%Treatment 2 & 0.0015681 & 0.910 \\
%Treatment 3 & 0.0009271 & 0.296 \\
%\hline
%\end{tabular}
%\end{table}

%% For two column figures and tables, use the following:

%% \begin{figure*}
%% \caption{Almost Sharp Front}\label{afoto}
%% \end{figure*}

%% \begin{table*}
%% \caption{Repeat length of longer allele by age of onset class}
%% \begin{tabular}{ccc}
%% table text
%% \end{tabular}
%% \end{table*}

%----------------------------------------------------------------------------------------

\end{document}